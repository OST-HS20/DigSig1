\section{Übertragungsfunktion}
\begin{center}
	\includegraphics[width=\columnwidth]{Images/übertraungsfunktion}
\end{center}

\textbf{Hinweis}:
\[
h[n] = [1, 2, 1] \transform H(z) = 1 + 2z^{-1} + z^{-2}
\]

\subsection{Delay}
Die Funktion $d(\omega)$ zeigt wieviele Verzögerung in Samples jede Frequenz hat. Ein Negatives vorzeichen deutet auf die positive Zeitverzögerung hin.
\[
d(\omega) = -\frac{\arg H(\omega)}{\omega} \xRightarrow{} \arg H(\omega) = -\omega d(\omega)
\]

Wenn ein Interval von Frequenzen eine gemeinsame Verzögerung hat, wird diese als \textbf{Group Delay} bezeichnet und kann durch Ableiten von $d(\omega)$ bestimmt werden. Sobald diese Konstant ist, ist das Intervall gefunden.
\[
d_g(\omega) = - \frac{d}{d\omega}\arg H(\omega)
\]

Wenn alle Frequenzen die gleiche Verzögerung haben ($d(\omega) = konst$) dann spricht man von einem \textbf{Linear Phase Filter}. Bei diesem das Zentrum der symetrische UTF nicht bei 0 sondern mit $D = d(\omega)$ Samples verzögert.
\begin{center}
	\includegraphics[width=0.5\columnwidth]{Images/verzögerung}
\end{center}

\textbf{Eigenschaften} von Linear Phasen Filter sind:
\begin{itemize}
	\item symmetrische Impulsantworten (rein reelö) haben keine Phase (Nullphasig) und damit keinen Delay
	\item durch das einführen einer Zeitverzögerung (um das Filter kaus zu machen) wird aus dem Nullphasigen Filter ein linearphasiges Filter.
\end{itemize}